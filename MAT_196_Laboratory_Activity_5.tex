\documentclass[10pt]{article}
\usepackage[utf8]{inputenc}
\usepackage[T1]{fontenc}
\usepackage{amsmath}
\usepackage{amsfonts}
\usepackage{amssymb}
\usepackage[version=4]{mhchem}
\usepackage{stmaryrd}
\usepackage{graphicx}
\usepackage[export]{adjustbox}
\graphicspath{ {./images/} }

\begin{document}

\section*{Model Formulation}

The system of equations is given by:
\[
\left\{
\begin{aligned}
S_{t+1} &= (1-p)S_t - \frac{\beta}{N}I_t S_t + b(R_t + I_t), \\
I_{t+1} &= \frac{\beta}{N}I_t S_t + (1-b-\gamma)I_t, \\
R_{t+1} &= (1-b)R_t + \gamma I_t + pS_t,
\end{aligned}
\right.
\]
where \( 0 < p+b < 1 \) and \( 0 < b + \gamma < 1 \). The parameters are defined as:
\begin{itemize}
    \item \( b \): probability of birth,
    \item \( \gamma \): probability of recovery,
    \item \( p \): proportion vaccinated,
    \item \( \beta \): contact rate,
    \item \( N \): total population size.
\end{itemize}
Since \( N \) is constant, then:
\[
N = S_t + I_t + R_t.
\]

\section*{Equilibrium Analysis}

Simplifying under the assumption \( N = S_t + I_t + R_t \), we eliminate \( R_t \). Thus:
\[
\left\{
\begin{aligned}
s_{t+1} &= (1-p)s_t - \frac{\beta}{N}I_t s_t + b(N - I_t - s_t + I_t), \\
I_{t+1} &= \frac{\beta}{N}I_t s_t + (1-b-\gamma)I_t.
\end{aligned}
\right.
\]

Simplifying further:
\[
\left\{
\begin{aligned}
s_{t+1} &= (1-p)s_t - \frac{\beta}{N}I_t s_t + b(N - s_t), \\
I_{t+1} &= \frac{\beta}{N}s_t I_t + (1-b-\gamma)I_t.
\end{aligned}
\right.
\]

### Equilibrium Conditions
The equilibrium conditions \( f(\bar{s}, \bar{I}) = 0 \) and \( g(\bar{s}, \bar{I}) = 0 \) yield:
\[
\begin{aligned}
(1-p)\bar{s} - \frac{\beta}{N}\bar{I}\bar{s} + b(N - \bar{s}) &= 0, \\
\frac{\beta}{N}\bar{I}\bar{s} + (1-b-\gamma)\bar{I} &= 0.
\end{aligned}
\]

From the second equation:
\[
\bar{I} \left( \frac{\beta}{N} \bar{s} + 1 - b - \gamma \right) = 0.
\]
Thus, either \( \bar{I} = 0 \) (disease-free equilibrium) or:
\[
\frac{\beta}{N} \bar{s} = b + \gamma.
\]

For \( \bar{I} = 0 \), substituting into the first equation:
\[
(1-p)\bar{s} + b(N - \bar{s}) = 0.
\]
Simplifying gives:
\[
\bar{s} = \frac{bN}{p + b}.
\]

For the endemic equilibrium (\( \bar{I} \neq 0 \)):
\[
\bar{s} = \frac{(b + \gamma)N}{\beta}.
\]

### Basic Reproduction Number
The basic reproduction number \( R_0 \) is defined as:
\[
R_0 = \frac{\beta b}{(p + b)(b + \gamma)}.
\]
If \( R_0 < 1 \), the disease will die out; otherwise, an epidemic occurs.

\section*{Jacobian Matrix}

The Jacobian matrix of the system is:
\[
J =
\begin{pmatrix}
1-p - \frac{\beta}{N}I - b & -\frac{\beta}{N}s \\
\frac{\beta}{N}I & \frac{\beta}{N}s + 1 - b - \gamma
\end{pmatrix}.
\]

For the disease-free equilibrium (\( \bar{I} = 0 \), \( \bar{s} = \frac{bN}{p+b} \)), the Jacobian becomes:
\[
J_1 =
\begin{pmatrix}
1-p-b & -\frac{\beta b}{p+b} \\
0 & \frac{\beta b}{p+b} + 1 - b - \gamma
\end{pmatrix}.
\]

The eigenvalues of \( J_1 \) are:
\[
\lambda_1 = 1-p-b, \quad \lambda_2 = \frac{\beta b}{p+b} + 1 - b - \gamma.
\]

The disease-free equilibrium is locally asymptotically stable (LAS) if:
\[
|\lambda_1| < 1 \quad \text{and} \quad |\lambda_2| < 1.
\]

For the endemic equilibrium (\( \bar{s} = \frac{(b+\gamma)N}{\beta} \)), substituting into \( J \), we analyze its stability similarly.

\section*{Conclusion}
The basic reproduction number \( R_0 \) determines the threshold for the outbreak of an epidemic. If \( R_0 < 1 \), the disease-free equilibrium is stable, and no epidemic occurs.

\end{document}
